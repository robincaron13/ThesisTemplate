\documentclass[a4paper, 11pt]{book}

\usepackage[utf8]{inputenc}
%\usepackage[frenchb]{babel}
\usepackage[english]{babel}

\usepackage{lipsum}
\usepackage{amsmath}
\usepackage{graphicx}
\usepackage{geometry}
\usepackage{mathtools}
\usepackage{slashed}
\geometry{hmargin=3cm, vmargin=3cm}

\usepackage{lmodern}
\usepackage{ae,aecompl}
\usepackage{graphicx}
\usepackage{eso-pic}	% Nécessaire pour mettre des images en arrière plan
\usepackage{array}		% Permet d'écrite 'THESE' de haut en bas
\usepackage[absolute,overlay]{textpos}
\usepackage[english]{minitoc}  %permet de faire une table des matieres par chapitre
\usepackage[a-1b]{pdfx}
\usepackage{hyperref}
\usepackage{multicol}
\usepackage{afterpage}

% ajoute (entre autre) la bibliographie dans la table des matieres 
\usepackage[nottoc]{tocbibind}

% biblio ordonnee classique
\bibliographystyle{unsrt}

\date{\today}

% \input{Pagedegarde}

% \author{Robin \textsc{Caron}}
% \title{\Huge{\textbf{Charm and beauty anisotropic flow in ultra-relativistic heavy-ions collisions with ALICE at the LHC}}}
% \ED{\'{E}cole doctorale \no 576 : Particles, Hadrons, Energy, Nuclei, Instrumentation, Imaging, Cosmos et Simulation (PHENIICS)}
% \doctorat{Doctorat Paris-Sud}
% \specialite{Physique Hadronique}
% \directeur{Javier \textsc{Castillo}}
% \encadrant{Stefano \textsc{Panebianco}}
% \date{20 juillet 2021}
% \jurya{M. Vincent Vega}{Professeur, MINES ParisTech}{Rapporteur}
% \juryb{M. Jules Winnfield}{Professeur, Arts Et Métiers ParisTech}{Rapporteur}
% \juryc{M. Butch Coolidge}{Chargé de recherche, ENS Cachan}{Examinateur}
% \juryd{Mme. Mya Wallace}{Danseuse, en freelance}{Examinateur}
% \jurye{M. Marsellus Wallace}{Ingénieur, MIT}{Examinateur}
% \ecole{l'Université Paris-Sud}
% \adresse{
% 	\textbf{Institut de Recherche sur les Lois Fondamentales de l'Univers \\ CEA Saclay - Département de Physique Nucléaire \\	91272 Gif-sur-Yvette, Essone, France}
% }






\begin{document}

\include{1ere}

%\pagedegarde
%\maketitle
\dominitoc

\chapter*{Abstract}
% pour faire apparaitre l'introduction dans le sommaire et que les minitocs soient au bon
% endroit
%\addstarredchapter{Abstract} 


\lipsum[22-26]

\chapter*{Résumé}
% pour faire apparaitre l'introduction dans le sommaire et que les minitocs soient au bon
% endroit
%\addstarredchapter{Résumé} 

\lipsum[22-26]

\chapter*{Remerciements}
% pour faire apparaitre l'introduction dans le sommaire et que les minitocs soient au bon
% endroit
%\addstarredchapter{Remerciements} 

\lipsum[22-26]

\chapter*{List of publications}
% pour faire apparaitre l'introduction dans le sommaire et que les minitocs soient au bon
% endroit
%\addstarredchapter{List of publications} 

\lipsum[22-24]


\tableofcontents

\chapter*{Recreating the primordial univers ?}
% pour faire apparaitre l'introduction dans le sommaire et que les minitocs soient au bon
% endroit
\addstarredchapter{Recreating the primordial univers ?} 

% Pour que l'entete soit correcte car chapter* ne redefinit pas l'entete.
\markboth{INTRODUCTION}{}


\lipsum[1-5]



\chapter{Ultra-relativistic heavy-ion collisions}


\minitoc

\section{Exploring matter under extreme conditions}

\lipsum[2-4]

\section{Current understanding of heavy-ion collisions}

\lipsum[4-6]

\section{Anisotropic flow as probe of QGP}

\lipsum[8-10]

\section{Quarkonia dissociation as signature of deconfinement}


\lipsum[2-10]


%\section{Dynamique des quarks et des gluons}
%\subsection{Suppresion séquentielle}


\chapter{Analysis techniques}

\minitoc


\lipsum[13-19]




\chapter{Experimental setup}

\minitoc


\lipsum[13-20]



\chapter{Processing the data and selection}

\minitoc


\lipsum[13-30]


\chapter{Data analysis}

\minitoc


\lipsum[13-40]



\chapter{Results and discussions}

\minitoc


\lipsum[13-30]



\include{conclu}

\appendix

\chapter{Systematic uncertainties}

\lipsum[26-27]


\chapter{Simulation Monte Carlo}

\lipsum[26-27]


\chapter{Calcul du flot}

\lipsum[26-27]



\bibliography{allbiblio}

\include{4eme}

\end{document}

